\documentclass[a4paper,12pt]{article}
% a4paper sets it up with the measures for A4 paper
% 12pt is the lettering size, (you can use 11pt or 12pt)

% this produces A4 pages with a lot more text on each page
\usepackage{a4wide}

% this tells LaTeX how the latex file is encoded
% on a modern linux usually utf8, older linux in Sweden usually latin1
% Windows has sometimes, but not always, its own encoding
% latin1 and utf8 allow direct input of accented characters like åäö 
\usepackage[utf8]{inputenc}


% uncomment (i.e. remove the % in front) for a fancier bibliography
%\bibliographystyle{natbib}

% this defines the \includegraphics command used for including figures
\usepackage{graphicx}

% the amsmath package has a lot of useful stuff for equations,
% especially for ligning up multiple equations
\usepackage{amsmath}

%useful for dealing with labels, uncomment while writing
%\usepackage{showkeys}


% if you prefer the equations numbered consecutively comment out
% the following line AND all the
% \setcounter{equation}{0} after every \section command
\renewcommand{\theequation}{\thesection.\arabic{equation}}

\begin{document}

%%%%%%%%%%%%%%%%%%%%%%%%%%%%%%%%%%%%%%%%%%%%%%%%%%%%%%%%%%%%%%%%%%%%%%%%%%%
% first as you might guess the title page
\begin{titlepage}
% flushright puts it towards the right of the page
%\begin{flushright}
% our preprint number yy-nn (year-number), ask your supervisor how to get this
%LU TP 14-nn\\
% some indication of the date
%June 2014\\
%\end{flushright}
\vfill
\begin{center}
% put in line breaks to make the title look nicer and provide more
% space between the lines
{\large\bf Genetic aetiology of the persistently lean phenotype, exposed to an unhealthy environment.}
\\[3cm]
{\bf Gunnar Käll\'en}
\\[5mm]
{Department of Astronomy and Theoretical Physics, Lund University}
\\[2cm]
{Bachelor/Master thesis supervised by Johan Bijnens}
\vfill
% use the eps version for latex and the pdf version for pdflatex 
\includegraphics[height=4cm]{logocLUeng.eps}
%\includegraphics[height=4cm]{logocLUeng.pdf}
\end{center}
\end{titlepage}
%%%%%%%%%%%%%%%%%%%%%%%%%%%%%%%%%%%%%%%%%%%%%%%%%%%%%%%%%%%%%%%%%%%%%%%%%%%
% a page with the abstract and popular description
\thispagestyle{empty} % do not count pages just yet

% these 'rubber' spaces make the page layout look nicer
%the \phantom{p} is to make something there for the \vfill to push against
\phantom{p}
\vfill

\section*{Abstract}
The abstract is a short summary describing the content of the main text.
This should give enough information about the contents to decide for the
intended audience whether further reading will be useful. The size should
be about half a page, best written at the end, after most of the thesis is
written.
This is a \LaTeX\ template for a bachelor/master thesis in theoretical physics
at Lund University. It contains examples of most needed structures
and discusses other relevant aspects of writing a thesis.

\vfill

\section*{Populärvetenskapligt sammanfattning}
% Swedish letters: \"a, \"o, \aa or ä, ö, å if you use utf8

This is meant to be popular \emph{introduction} to and
\emph{description} of your thesis, preferably
written in Swedish. The name is unfortunately misleading. It is not a summary
but mainly an introduction to what you have done. A good idea is to write this
when you are about one third through the time allotted for the thesis work.

Especially important here are the context of your project and why this is
an interesting project to do. This should be about half a page as well.

\vfill

\newpage

\tableofcontents
% the list of figures and tables is optional
\listoffigures 
\listoftables

\newpage

\section{Introduction}
\setcounter{equation}{0}
\label{sec:introduction}


This is a template \LaTeX\ file for a bachelor or master thesis in theoretical
physics at Lund University. It has an example front page and in the \LaTeX\ file
itself there are many comments. The introduction should contain the background
of why you are doing this and a short description of the general area.

Writing a thesis is a time consuming project. Especially in theoretical physics
where a bachelor/master thesis typically contains a large number of equations.
These are very difficult to handle using more standard programs or
applications.

\section{\LaTeX}
\setcounter{equation}{0}

There are very many introductions to \LaTeX\ around. A simple one is
``A not-so-short introduction to \LaTeX'' \cite{notsoshort}.
Inclusion of graphics is done with the package {\tt graphicx} \cite{graphicx}.
The sections below will contain examples of most of the things that might be
needed in writing a thesis and a number of hints and comments.
Pay especially heed to the latter. you will be grateful when hunting for
typing errors and other mistakes later. Before reading the remainder
of this template, make sure you have a basic knowledge of \LaTeX.

\section{Comments}
\setcounter{equation}{0}

The comments below are done with a simple listing environment called
{\tt itemize}.
\begin{itemize}
\item
In the template file itself there are many comments as well, it will probably
pay off to look at those as well.
\item
Type cleanly, especially equations. It will make life easier hunting for errors.
\item
Both \TeX\ and \LaTeX\ themselves classify as a full programming language, so
you can do some really fancy stuff with it. However, if not needed, it can
lead to very funny behaviour if you don't exactly know what the funny
commands do and a small change messes up things.
\item
You probably already have, but have a look at earlier bachelor or master 
theses, for examples.
\item
Examples of useful things all over in the file, look at them.
\end{itemize}

\section{Labeling}
\setcounter{equation}{0}
\label{sec:labeling}


Latex is very nice because it can help you to refer to the right table, figure,
item section orpage. Just remember that you need to compile with latex or
pdflatex twice (!) before it has updated all the cross references. This is also
true for references (see Sect.\,\ref{sect:refs}). If you use some special
packages, sometimes it requires even more times processing with latex
or pdflatex.

In item\,\ref{label:item} on page \pageref{label:item} you can find some
important information that was not covered in Eq.\,(\ref{eq:smart}) or in
Table\,\ref{table:1}. That covers the most important references\footnote{If you
want to learn more -- read more.}

\LaTeX\ allows to use labels for references, tables, sections,\ldots.
Use this consistently. If you do not, you will have to go through the entire
manuscript by hand if you add an equation or a figure and change all the
numbers. You will for sure forget some. Add the command
{\tt {$\backslash$}usepackage\{showkeys\}} at the top of the file. It will make
labels and where they are used visible which is very useful while writing.
The labels can be (almost) anything but choosing them wisely and easier
to remember helps very much.

The actual referring can be done in many ways. The words equation,
tabke, figure and section can be included or not, as well as capitalized
or not. There is no preferred way of doing this but be consistent.
Use the same style throughout.
Citations to the literature are in our field put at the end and the most common
style is ordering by appearance and a number in square brackets like
\cite{Kallen:1952zz}. That, by the way, is the real reference for the
first part of the title.

You can refer to a section as Sect.~\ref{sec:introduction}
and a subsection as Sect.~\ref{sec:asubsection}.


\subsection{A subsection of Section 1}
\label{sec:asubsection}

\subsubsection{A subsubsections}

\paragraph{Paragraphs are not numbered}

Paragraphs are allowed but usually not used.

\section{An example computer code}
\setcounter{equation}{0}

When we write down computer code we might
want it to look just like it does in the editor (using a fixed with font). This
is done using the {\sl verbatim} environment.

\begin{verbatim}

       PROGRAM myfortran

       IMPLICIT NONE

       REAL*8 mag(20)
       REAL flux(20)
       INTEGER nstar

       WRITE(*,*) "This program calculates a magnitude"
       READ(*,*) flux(1)
       mag(1)=-2.5*LOG(flux(1))

\end{verbatim}

Computer languages are, usually but not always, written in the smallcaps font
like {\sc FORTRAN}.


\section{Figures}
\setcounter{equation}{0}

Figures are of course very important in the thesis. Make sure that your figures
have thick lines that stand out in the printed version, that the axes are
labelled and explained and that colours are distinguishable in a black and
white printout as well (this is helpful for the not insignificant fraction of
the population who are colorblind).
\begin{figure}
%\includegraphics[width=8cm]{Draco_cmd.ps}
\caption[CMD of Draco]{This colour-magnitude diagram shows the giant branch of
the Draco dwarf spheroidal galaxy as seen in the Str\"omgren filter system
(also known as $uvby$). Are the lines thick enough to read? Did you find out
how to make nice postscript fonts with your plotting program?}
\end{figure}

\section{Tables}
\setcounter{equation}{0}

This section contains a table which shows the most basic elements and a simple
layout. The table can be seen in Table\,\ref{table:1}.  Make sure to make your
labeling system easy. Maybe table:1 is not so smart -- what if you move the
table somewhere else, and it is not any longer the first table or you add a
table before this one. Perhaps a label like {\tt table:varstars} would have
been better?
\begin{table}[!h]
\caption{This is a table of variable stars}\smallskip
\label{table:1}
\centering  
\begin{tabular}{lrrc}
\hline\hline  
\smallskip
Id of star & I &  V & Var.? \\
\hline
1234 & 15.6 & 17.3 & No \\
5677 & 13.4 & 12.3 & Yes\\
\hline
\end{tabular}
\end{table}

Here we managed to place the table directly in the text (using the !h option).
Generally we should let latex control the positions of figures and tables --
if you are unhappy with their placement then try to move them around in the raw
text or experiment with !h and !t.

\section{Equations}
\setcounter{equation}{0}

Writing mathematical formular in latex is not always so easy at first. But it
does look good! There are several environments that we can use, and we can get
numbers for all the equations etc. Very neat. If I want simply to have a small
equation or some expression in the text I can just do {\tt \$ x+a  cdot b= f(x)
\$}. Which, when latexed, gives us the formula $x+a \cdot b= f(x)$.
\\ \\
If we want an equation by itself, we just add one more {\tt \$} at each side:
$$ g(x,y)=\sin (x) + 10 \ log(y \cdot 20 \cdot 10^{-2x}) $$
It is also possible to use an {\sl environment} especially for equations.
Remember that equations should be integrated in the flow of the text, even if
they are on separate lines. Therefore we should use punctuation in equations as
well, for example in the equation which reads
\begin{equation}
  \sum_{k=i}^n (x- \overline{x})^2 +\sqrt{x^2 + y^3} = h(x,y) \label{eq:smart}
  \, .
\end{equation}

Remember that the {\tt equation} environment does not like empty lines.  This
is the same in {\tt tabular}, by the way.  You can do many more things in the
maths-mode. If you are going to write lots of equations you will learn it very
quickly.

You should also look a the American Mathemematical Society package
{\tt amsmath} \cite{amsmath}. It is very useful for lining up equations and
has a lot more symbols than the base \LaTeX\ set.

Equations are typeset in a separate font, but note that units,
abbreviations and names should not be typeset in this font. This way
we can distinguish m for meter from $m$ for mass. Inside an equation
the $\backslash$mathrm\{\ldots\} command can be used to typeset in the normal font,
e.g. $m_{\mathrm{eff}}$ rather than $m_{eff}$ (unless $m$ is a tensor
with three indices). 

\section{The list environment}
\setcounter{equation}{0}

If you want to make good looking lists, short or long, latex can do it for you.

\begin{itemize}
\item This is the first entry
\item and the second one
  \begin{itemize}
    \item lets go down one level
    \item and stay there
      \begin{itemize}
          \item And one more
          \item very deep
      \end{itemize}
  \end{itemize}
\end{itemize}

But you can also do other types of lists. For example with numbers. Very useful
as you can refer to them later with labels.

\begin{enumerate}
  \item First entry
  \item Second entry
  \item Third entry
    \begin{enumerate} 
  \item First entry \label{label:item}
  \item Second entry
  \end{enumerate}
      \begin{itemize}
          \item And you can mix the listings
          \item like this 
      \end{itemize}
\end{enumerate}     


\section{References}
\setcounter{equation}{0}
\label{sect:refs}

You also need to cite all the thick and good papers that you have read during
your thesis work. There are several ways. Here we will use the 
standard style given by \LaTeX\ 
as that one is the one that most resembles the way we write references in
high energy physics journal papers. 

You can add the references in a so called bibtex file. This file contains all
the information latex needs about a single paper in order to make an entry in
the reference list and to write the correct reference inside your text. 
Alternatively we can add a list of bibitems directly in the latex file. This
is what we will do here, and probably the most useful for you.

Finding references can be done by many means, Google, your supervisor, etc..
In particle physics there is a very good database called INSPIRE
\cite{inspire}. They also allow to get the \LaTeX\ code for a reference
directly. We use the style Latex EU, see \cite{Kallen:1952zz} in the references
how their output looks like.

\section{Conclusions}
\setcounter{equation}{0}

\section*{Acknowledgements}

There is no acknowledgements section in the regular latex, but you can easily
make one yourself. The * makes in the command makes it not have a number

% this changes the numbering of sections to letters as is needed
% for an appendix
\appendix

\section{This is an appendix}
\setcounter{equation}{0}

You can put long mathematical derivations or tables in appendices.
\begin{equation}
\label{appendixequation}
x = y\,.
\end{equation}

\section{This is another appendix}
\setcounter{equation}{0}

Subsections etc. are also allowed here

\subsection{An appendix subsection}


%%%%%%%%%%%%%%%%%%%%%%%%%%%%%%%%%%%%%%%%%%%%%%%%%%%%%%%%%%%%%%%%%%%%%%%%%%
\begin{thebibliography}{99}
\bibitem{notsoshort}
T. Oetiker {it et al.}, ``The Not So Short Introduction to \LaTeX2e,''
{\tt http://www.ctan.org/tex-archive/info/lshort/english/}

\bibitem{graphicx}
{\tt http://ctan.org/pkg/graphicx}

\bibitem{Kallen:1952zz}
G.~Kallen,
%``On the definition of the Renormalization Constants in Quantum Electrodynamics,''
Helv.\ Phys.\ Acta {\bf 25} (1952) 417.
%%CITATION = HPACA,25,417;%%

\bibitem{amsmath}
``User's guide for the {\tt amsmath} package,''
{\tt http://www.ctan.org/pkg/amsmath}

\bibitem{inspire} High-Energy Physics Literature Database,
{\tt http://inspirehep.net}

\end{thebibliography}


\end{document}
