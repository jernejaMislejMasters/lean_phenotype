\documentclass[a4paper,12pt]{article}
% a4paper sets it up with the measures for A4 paper
% 12pt is the lettering size, (you can use 11pt or 12pt)

% this produces A4 pages with a lot more text on each page
\usepackage{a4wide}

% this tells LaTeX how the latex file is encoded
% on a modern linux usually utf8, older linux in Sweden usually latin1
% Windows has sometimes, but not always, its own encoding
% latin1 and utf8 allow direct input of accented characters like åäö 
\usepackage[utf8]{inputenc}


% uncomment (i.e. remove the % in front) for a fancier bibliography
%\bibliographystyle{natbib}

% this defines the \includegraphics command used for including figures
\usepackage{graphicx}

% the amsmath package has a lot of useful stuff for equations,
% especially for ligning up multiple equations
\usepackage{amsmath}

%useful for dealing with labels, uncomment while writing
%\usepackage{showkeys}


% if you prefer the equations numbered consecutively comment out
% the following line AND all the
% \setcounter{equation}{0} after every \section command
\renewcommand{\theequation}{\thesection.\arabic{equation}}

\begin{document}

%%%%%%%%%%%%%%%%%%%%%%%%%%%%%%%%%%%%%%%%%%%%%%%%%%%%%%%%%%%%%%%%%%%%%%%%%%%
% first as you might guess the title page
\begin{titlepage}
% flushright puts it towards the right of the page
%\begin{flushright}
% our preprint number yy-nn (year-number), ask your supervisor how to get this
%LU TP 14-nn\\
% some indication of the date
%June 2014\\
%\end{flushright}
\vfill
\begin{center}
% put in line breaks to make the title look nicer and provide more
% space between the lines
{\large\bf Genetic aetiology of the persistently lean phenotype, exposed to an unhealthy environment.}
\\[3cm]
{\bf Jerneja Mislej}
\\[5mm]
\\[2cm]
{Master thesis supervised by Alaitz Poveda}
\\[2cm]
% use the eps version for latex and the pdf version for pdflatex 
\includegraphics[height=4cm]{Lund_University_logotype.png}
%\includegraphics[height=4cm]{logocLUeng.pdf}
{\\\\2017}
\\[5cm]
\end{center}
{\\Examensarbete för 30 hp\\
Institutionen för ?, ? fakulteten, Lunds universitet
\\
\\Thesis for a ? in ?, 30 ECTS credits
\\Department of ?, Faculty of ?, Lund University}

\end{titlepage}
%%%%%%%%%%%%%%%%%%%%%%%%%%%%%%%%%%%%%%%%%%%%%%%%%%%%%%%%%%%%%%%%%%%%%%%%%%%
% a page with the abstract and popular description
\thispagestyle{empty} % do not count pages just yet

% these 'rubber' spaces make the page layout look nicer
%the \phantom{p} is to make something there for the \vfill to push against

\section*{Abstract}
Abstract goes here, formated in text....

\newpage

\phantom{p}
\vfill

\newpage

\tableofcontents
% the list of figures and tables is optional
\listoffigures 
\listoftables

\newpage

\section{Introduction}
\setcounter{equation}{0}



Introduction goes here, formated in text....

\section{Methods}
\setcounter{equation}{0}

Methods go here, formated in text....

\section{Results}
\setcounter{equation}{0}

Results go here, formated in text....

\section{Discussion}
\setcounter{equation}{0}

Discussion goes here, formated in text....


\section{Conclusions}
\setcounter{equation}{0}

Conclusions go here, formated in text....

% this changes the numbering of sections to letters as is needed
% for an appendix
\appendix

\section{This is an appendix}
\setcounter{equation}{0}

You can put long mathematical derivations or tables in appendices.
\begin{equation}
\label{appendixequation}
x = y\,.
\end{equation}

\section{This is another appendix}
\setcounter{equation}{0}

Subsections etc. are also allowed here

\subsection{An appendix subsection}


%%%%%%%%%%%%%%%%%%%%%%%%%%%%%%%%%%%%%%%%%%%%%%%%%%%%%%%%%%%%%%%%%%%%%%%%%%
\begin{thebibliography}{99}
\bibitem{notsoshort}
T. Oetiker {it et al.}, ``The Not So Short Introduction to \LaTeX2e,''
{\tt http://www.ctan.org/tex-archive/info/lshort/english/}

\bibitem{graphicx}
{\tt http://ctan.org/pkg/graphicx}

\bibitem{Kallen:1952zz}
G.~Kallen,
%``On the definition of the Renormalization Constants in Quantum Electrodynamics,''
Helv.\ Phys.\ Acta {\bf 25} (1952) 417.
%%CITATION = HPACA,25,417;%%

\bibitem{amsmath}
``User's guide for the {\tt amsmath} package,''
{\tt http://www.ctan.org/pkg/amsmath}

\bibitem{inspire} High-Energy Physics Literature Database,
{\tt http://inspirehep.net}

\end{thebibliography}


\end{document}
